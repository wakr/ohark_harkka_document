\documentclass{tktltiki}
\usepackage[pdftex]{graphicx}
\usepackage{subfigure}
\usepackage{url}
\begin{document}
%\doublespacing
%\singlespacing
\onehalfspacing

\title{Ruby on Rails -sovelluskehys: case: Translator}
\author{Kristian Wahlroos - 014417003}
\date{\today}

\maketitle

\numberofpagesinformation{\numberofpages\ sivua + \numberofappendixpages\ liitesivua}
\classification{\protect{\ \\
A.1 [Introductory and Survey],\\
I.7.m [Document and text processing]}}

\keywords{ulkoasu, l�hdeluettelo}


\mytableofcontents

% PDF --> BIB --> PDF --> PDF -- VIEW

\section{Johdanto}
Hello \cite{erkio01}.
\newpage

\section{Arkkitehtuurityylien perusideat ja roolijako}
Roolijako here
\newpage

\section{Yleisarkkitehtuuri ja keskeisimm�t variaatiopisteet}

Luokkakaavio, sekvenssikaavio, stereotyypit.

\subsection{Yleiskuva kehysrakenteesta}
\subsection{Kehyksen erikoistaminen sovelluskohtaisesti}
\subsection{Suunnittelumallit}
\subsection{Esimerkkitapaukset}
\newpage


\section{Kehyksen ja sovelluksen arviointi}
\subsection{Hyv�t ja huonot puolet}
L�hdekirjallisuus
\subsection{Laatuskenaariot}
\subsection{ATAM}



\newpage
\cleardoublepage
%
% Sitten alkaa l�hdeluettelo
%


\nocite{*}
\bibliographystyle{tktl}
\bibliography{lahteet}

\lastpage

\appendices

\pagestyle{empty}

\end{document}
